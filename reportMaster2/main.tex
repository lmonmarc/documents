\documentclass[11pt,onecolumn]{report}
\usepackage[utf8]{inputenc}
\usepackage[T1]{fontenc}

\usepackage[french]{babel}

\usepackage{graphicx}
\graphicspath{{./img/}}

\usepackage{geometry}
\geometry{hmargin=3cm,vmargin=3cm}

\usepackage{tikz}
\usepackage{enumitem}
\usepackage{subcaption}
\usepackage{amsmath}

\usepackage{hyperref}
\hypersetup{colorlinks=true}


\begin{document}

\begin{titlepage}

  \begin{center}
    \begin{figure}[!htbp]
      \begin{center}
        \begin{minipage}{0.4\linewidth}
          \center{\includegraphics[width=1\textwidth]{enseirb_inp.png}}
        \end{minipage}
        \hfill
        \begin{minipage}{0.4\linewidth}
          \center{\includegraphics[width=1\textwidth]{UniversiteBordeaux_CMJN-01.jpg}}
        \end{minipage}
      \end{center}
    \end{figure}
  \end{center}

  \vspace{1cm}
  \hrule
  \begin{flushleft}
    \Huge{\textit{Exemple d'utilisation de l'interface SYCL}}\\
  \end{flushleft}
  \begin{flushright}
    \huge\textbf{Rapport de stage}\\
  \end{flushright}
  \hrule

  \vspace{1cm}
  \center{\large {Alexandre Honorat}}\\
  \vspace{1cm}
  \large{\textbf{Encadrants :} Olivier Aumage, Denis Barthou}\\
  \large{\textbf{Tuteur :} Denis Barthou}
  \vspace{1cm}

  \center{\small{Filière Informatique, spécialité Parallélisme, Régulation et Calcul Distribué\\Master Réseaux et Systèmes Mobiles, parcours Calcul Haute Performance\\\today}}

  \vspace{1cm}

  \begin{center}
    \begin{figure}[!htbp]
      \begin{center}
        \begin{minipage}{0.4\linewidth}
          \center{\includegraphics[width=1\textwidth]{labri.jpeg}}
        \end{minipage}
        \hfill
        \begin{minipage}{0.4\linewidth}
          \center{\includegraphics[width=1\textwidth]{INRIA_CORPO_CMJN.png}}
        \end{minipage}
      \end{center}
    \end{figure}
  \end{center}

\end{titlepage}


\tableofcontents
%% Intro:
%% -> contexte //nécessité de la programmation parallèle
%% -> problématique //dur d'écrire du code parallèle (même pour le compilateur)
%% -> contribution //avoir de l'expressivité, exemple entre QIRAL et SYCL pour exécuter des stencils

%% Chap2:
%% -> contexte
%% -> état de l'art
%% -> problématique

%% Chap3:
%% -> proposition (aspect théorique)

%% Chap4:
%% -> mise en \oe uvre, éléments d'implantation
%%    i. e. C++, SYCL

%% Chap5:
%% -> évaluation, méthodologie présentation résultat

%% Conclusion:
%% -> ce qui a été fait (résumé)
%% -> ce qui aurait pu être fait (qqc de crédible, choix alternatifs par exemple SYCLONE, SkePU, StarPU)
%% -> ce qui reste à faire (à court (plutôt technique), moyen et long terme) (aspect ditribué ? pour aute algo que stencils comme map/reduce ?)

\chapter*{Remerciements}

DB OA RK SP

\chapter{Introduction}
%\addcontentsline{toc}{chapter}{Introduction}

\section{Parallélisation d'algorithmes}

Nécessaire pour avoir des résultats en un temps acceptables.

\section{\'Ecriture de code parallèle}

C''est difficile, même pour le compilateur

\section{Choix du niveau d'expression du parallèlisme}

Où décrire le parallèlisme ? Le trouver automatiquement ?
cf Qiral et Maude


\chapter{Extraction et formalisation du parallélisme}

\section{Contexte}

\subsection{Expression du parallélisme}

Langages dédiés aux scientifiques Matlab, QIRAL, Matematica

Langages dédiés au parallèlisme Cilk, SkePU

Runtimes/bibliothèques dédiées au parallèlisme StarPU, Parsec, OpenCL, SYCL, Kokkos, Blitz++, Global Arrays

Classification des algorithmes articles Bones

\subsection{Cas d'application: les stencils}

Jacobi2D

\section{Conception d'un DSEL pour les stencils}

\subsection{Objectifs sémantiques}

De quoi a-t-on besoin pour décrire un stencil

-> indices coefs

-> pattern du stencil

-> tableaux coefs

-> tableaux données

-> abstraire les structurations de données (copies tuiles, et data layout)

\subsection{Objectifs de performances}

-> tuiles/caches (data reuse)

-> calculs à la compilation

\chapter{Bibliothèque de stencils basée sur SYCL}

\section{Présentation générale}

Fonctions disponibles

\section{Détails d'implantation}

C++14 et autres

\section{\'Evolutions possibles}

Reprendre slides : surtout agrégation de stencils



\chapter{\'Evaluation de la bibliothèque}


\section{\'Evaluation qualitative}

Comparaisons d'expressivités (nombre de lignes, nombres de symboles, typage à la compilation), factorisation de code
Reprendre slides : différents cas de tests


\section{Performances quantitatives sur machines dédiées}

Description du matériel

Mesures de temps

Mesures de compteurs

Vérification des valeurs numériques

\chapter{Conclusion}
%\addcontentsline{toc}{chapter}{Conclusion}

\section{Bilan}

Résumé du reste

Choix alternatifs SkePU, StarPU, SYCLONE


\section{Perspetives}

Intégration QIRAL et 4D, Aspects ditribués, MIC, layout, élargissements aux autres cas d'algorithmes (map/reduce)


\addcontentsline{toc}{chapter}{Bibliographie}
\nocite{*} %% à commenter
\bibliographystyle{plain}
\bibliography{main}


\end{document}
