\documentclass[11pt,onecolumn]{report}
\usepackage[utf8]{inputenc}
\usepackage[T1]{fontenc}

\usepackage[french]{babel}

\usepackage{graphicx}
\graphicspath{{./img/}}

\usepackage{geometry}
\geometry{hmargin=3cm,vmargin=3cm}

\usepackage{tikz}
\usepackage{enumitem}
\usepackage{subcaption}
\usepackage{amsmath}

\usepackage{hyperref}
\hypersetup{colorlinks=true}


\begin{document}

\begin{titlepage}

  \begin{center}
    \begin{figure}[!htbp]
      \begin{center}
        \begin{minipage}{0.4\linewidth}
          \center{\includegraphics[width=1\textwidth]{enseirb_inp.png}}
        \end{minipage}
        \hfill
        \begin{minipage}{0.4\linewidth}
          \center{\includegraphics[width=1\textwidth]{UniversiteBordeaux_CMJN-01.jpg}}
        \end{minipage}
      \end{center}
    \end{figure}
  \end{center}

  \vspace{1cm}
  \hrule
  \begin{flushleft}
    \Huge{\textit{Exemple d'utilisation de l'interface SYCL}}\\
  \end{flushleft}
  \begin{flushright}
    \huge\textbf{Rapport de stage}\\
  \end{flushright}
  \hrule

  \vspace{1cm}
  \center{\large {Alexandre Honorat}}\\
  \vspace{1cm}
  \large{\textbf{Encadrants :} Olivier Aumage, Denis Barthou}\\
  \large{\textbf{Tuteur :} Denis Barthou}
  \vspace{1cm}

  \center{\small{Filière Informatique, spécialité Parallélisme, Régulation et Calcul Distribué\\Master Réseaux et Systèmes Mobiles, parcours Calcul Haute Performance\\\today}}

  \vspace{1cm}

  \begin{center}
    \begin{figure}[!htbp]
      \begin{center}
        \begin{minipage}{0.4\linewidth}
          \center{\includegraphics[width=1\textwidth]{labri.jpeg}}
        \end{minipage}
        \hfill
        \begin{minipage}{0.4\linewidth}
          \center{\includegraphics[width=1\textwidth]{INRIA_CORPO_CMJN.png}}
        \end{minipage}
      \end{center}
    \end{figure}
  \end{center}

\end{titlepage}


\tableofcontents

\chapter*{Introduction}
\addcontentsline{toc}{chapter}{Introduction}

%% \chapter{Centre de recherche et cadre de travail}
%% \chapter{Bibliothèque de stencils basée sur SYCL}
%% \chapter{Utilisation de StarPU et intégration dans QIRAL}

%% Intro:
%% -> contexte //nécessité de la programmation parallèle
%% -> problématique //dur d'écrire du code parallèle (même pour le compilateur)
%% -> contribution //avoir de l'expressivité, exemple entre QIRAL et SYCL pour exécuter des stencils

%% Chap2:
%% -> contexte
%% -> état de l'art
%% -> problématique

%% Chap3:
%% -> proposition (aspect théorique)

%% Chap4:
%% -> mise en \oe uvre, éléments d'implantation
%%    i. e. C++, SYCL

%% Chap5:
%% -> évaluation, méthodologie présentation résultat

%% Conclusion:
%% -> ce qui a été fait (résumé)
%% -> ce qui aurait pu être fait (qqc de crédible, choix alternatifs par exemple SYCLONE, SkePU, StarPU)
%% -> ce qui reste à faire (à court (plutôt technique), moyen et long terme) (aspect ditribué ? pour aute algo que stencils comme map/reduce ?)

\chapter*{Conclusion}
\addcontentsline{toc}{chapter}{Conclusion}


\addcontentsline{toc}{chapter}{Bibliographie}
\nocite{*} %% à commenter
\bibliographystyle{plain}
\bibliography{main}


\end{document}
