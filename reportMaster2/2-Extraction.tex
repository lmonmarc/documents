%% -*- eval: (flyspell-mode 1); -*-

\chapter{Extraction et formalisation du parallélisme}

Le parallélisme dans un programme informatique est difficile à trouver automatiquement et la plupart du temps il est exprimé explicitement par le développeur. Il est alors nécessaire de lui offrir des outils permettant d'écrire le code parallèle simplement, en lui proposant notamment des fonctions et des mots-clés dans le langage choisi qui sont en accord à la fois avec les besoins matériels et à la fois avec les besoins algorithmiques. Ce chapitre dresse d'abord un état de l'art des principales solutions existantes, puis introduit les principaux objectifs du DSEL développé pendant le stage.

\section{Contexte}



\subsection{Expression du parallélisme}

Langages dédiés aux scientifiques Matlab, QIRAL, Matematica

Langages dédiés au parallèlisme Cilk, SkePU

Runtimes/bibliothèques dédiées au parallèlisme StarPU, Parsec, OpenCL, SYCL, Kokkos, Blitz++, Global Arrays

Classification des algorithmes articles Bones

\subsection{Cas d'application: les stencils}

Jacobi2D

\section{Conception d'un DSEL pour les stencils}

\subsection{Objectifs sémantiques}

De quoi a-t-on besoin pour décrire un stencil

-> indices coefs

-> pattern du stencil

-> tableaux coefs

-> tableaux données

-> abstraire les structurations de données (copies tuiles, et data layout)

\subsection{Objectifs de performances}

-> tuiles/caches (data reuse)

-> calculs à la compilation

